% Copyright © 2015 Lukas Rosenthaler, Benjamin Geer, Ivan Subotic,
% Tobias Schweizer, André Kilchenmann, and André Fatton.
%
% This file is part of Knora.
%
% Knora is free software: you can redistribute it and/or modify
% it under the terms of the GNU Affero General Public License as published
% by the Free Software Foundation, either version 3 of the License, or
% (at your option) any later version.
%
% Knora is distributed in the hope that it will be useful,
% but WITHOUT ANY WARRANTY; without even the implied warranty of
% MERCHANTABILITY or FITNESS FOR A PARTICULAR PURPOSE.  See the
% GNU Affero General Public License for more details.
%
% You should have received a copy of the GNU Affero General Public
% License along with Knora.  If not, see <http://www.gnu.org/licenses/>.

% Compile with XeLaTeX.

\documentclass[12pt, a4paper]{article}

\usepackage{listings}
\usepackage{fontspec}

\defaultfontfeatures{Scale=MatchLowercase,Mapping=tex-text}

\usepackage[dvipsnames]{xcolor}

\usepackage{polyglossia}
\setmainlanguage[variant=british]{english}
\setotherlanguage{french}
\setotherlanguage{german}

\usepackage[autostyle,english=british]{csquotes}

\usepackage[backend=biber]{biblatex}
\bibliography{bibliography}

\usepackage{tikz}
\usetikzlibrary{shapes.geometric} % required for the ellipse shape
\tikzset{vertex style/.style={
    draw=#1,
    thick,
    fill=#1!70,
    text=white,
    ellipse,
    minimum width=2cm,
    minimum height=0.75cm,
    font=\footnotesize,
    outer sep=3pt,
  },
}
\tikzset{literal style/.style={
    draw=#1,
    thick,
%    fill=#1!70,
    text=black,
    rectangle,
    minimum width=2cm,
    minimum height=0.75cm,
    font=\footnotesize,
    outer sep=3pt,
  },
}
\tikzset{
  text style/.style={
    sloped, % the text will be parallel to the connection 
    text=black,
    font=\tiny,
    above
  }
}

\usepackage{hyperref}
\usepackage[textwidth=3cm]{todonotes}
\newcommand{\issue}[2]{\todo{\href{https://github.com/dhlab-basel/Knora/issues/#1}{Issue #1: #2}}}

\title{The Knora Base Ontology}
\author{Lukas Rosenthaler, Benjamin Geer, Tobias Schweizer, Ivan Subotic\\Digital Humanities Lab, University of Basel}

\date{3 August 2016}

\begin{document}

\maketitle

\listoftodos

\tableofcontents

\section{Introduction}

\subsection{Resource Description Framework (RDF)}
Knora\footnote{{\bf Kn}owledge {\bf O}rganization, {\bf R}epresentation, and {\bf A}nnotation, \url{http://knora.org}.} uses a hierarchy of ontologies based on RDF \cite{RDF_Primer_1_1}, RDF Schema (RDFS)\cite{RDF_Schema_1_1}, and Web Ontology Language (OWL)\cite{OWL_2_Primer}. Both RDFS and OWL are expressed in RDF. RDF expresses information as a set of statements (called {\em triples}). A triple consists of a subject, a predicate, and an object (Figure \ref{fig:rdf-triple}).

\begin{figure}[h]
\centering
\begin{tikzpicture}
\node[vertex style=Blue] (sub) {subject};
\node[vertex style=Red, right of=sub,xshift=10em] (obj) {object}
edge [<-,cyan!60!blue] node[text style,above]{predicate} (sub);
\end{tikzpicture}
\caption{An RDF triple.}
\label{fig:rdf-triple}
\end{figure}

The object may be either a literal value (such as a name or number) or another subject. Thus it is possible to create complex graphs that connect many subjects, as in Figure \ref{fig:rdf-graph}.

\begin{figure}[h]
\centering
\begin{tikzpicture}
\node[vertex style=Blue] (sub1) {subject no. 1};

\node[literal style=Red, right of=sub1,xshift=10em,yshift=3em] (lit1) {literal no. 1}
edge [<-,cyan!60!blue] node[text style,above]{predicate no. 1} (sub1);

\node[literal style=Red, right of=sub1,xshift=10em] (lit2) {literal no. 2}
edge [<-,cyan!60!blue] node[text style,above]{predicate no. 2} (sub1);

\node[vertex style=Blue, right of=sub1,xshift=10em,yshift=-3em] (sub2) {subject no. 2}
edge [<-,cyan!60!green] node[text style,above]{predicate no. 3} (sub1);

\node[literal style=Red, right of=sub2,xshift=10em] (lit3) {literal no. 3}
edge [<-,cyan!60!blue] node[text style,above]{predicate no. 4} (sub2);

\node[vertex style=Blue, right of=sub2,xshift=10em,yshift=-3em] (sub3) {subject no. 3}
edge [<-,cyan!60!green] node[text style,above]{predicate no. 5} (sub2);

\end{tikzpicture}

\caption{An RDF graph.}
\label{fig:rdf-graph}
\end{figure}

RDF uses unique, URL-like identifiers called Internationalized Resource Identifiers (IRIs)\cite{RFC_3987}. Anything that can be a subject has an IRI, as does each predicate. Within a given project, IRIs typically differ only in their last component (the \enquote{local part}), which is often the fragment following a \texttt{\#} character. Such IRIs share a long \enquote{prefix}. In Turtle\cite{Turtle} and similar formats for writing RDF, a short prefix label can be defined to represent the long prefix. Then an IRI can be written as a prefix label and a local part, separated by a colon \enquote{\texttt{:}}. For example, if the \enquote{example} project's long prefix is \texttt{http://www.example.org/rdf\#}, and it contains subjects with IRIs like \texttt{http://www.example.org/rdf\#book}, we can define the prefix label \enquote{\texttt{ex}} to represent the prefix label, and write prefixed names for IRIs as in Figure \ref{fig:rdf-graph-with-prefix-labels}. In this document, we use the prefix label \texttt{kb} to represent the Knora base ontology,\footnote{\texttt{http://www.knora.org/ontology/knora-base\#}} but we usually omit it for brevity.

\begin{figure}[h]
\centering

\begin{tikzpicture}
\node[vertex style=Blue] (book) {ex:book1};

\node[literal style=Red, below of=book,xshift=-6em,yshift=-5em] (title) {\enquote{Das Narrenschiff}}
edge [<-,cyan!60!blue] node[text style,above]{ex:title} (book);

\node[literal style=Red, right of=book,xshift=6em,yshift=-5em] (author) {\enquote{Sebastian Brant}}
edge [<-,cyan!60!blue] node[text style,above]{ex:author} (book);

\node[vertex style=Blue, below of=book,xshift=0em,yshift=-8em] (page) {ex:page1}
edge [->,cyan!60!green] node[text style,above]{ex:pageOf} (book);

\node[literal style=Red, right of=page,xshift=10em,yshift=-3em] (pagename) {\enquote{a4r}}
edge [<-,cyan!60!blue] node[text style,above]{ex:pagename} (page);

%\node[vertex style=Blue, right of=sub2,xshift=10em,yshift=-3em] (sub3) {subject no. 3}
%edge [<-,cyan!60!green] node[text style,above]{predicate no. 5} (sub2);

\end{tikzpicture}

\caption{An RDF graph written with prefix labels.}
\label{fig:rdf-graph-with-prefix-labels}
\end{figure}

\section{The Knora Data Model}

\label{sec:data-model}

The Knora data model is based on the observation that, in the humanities, a value or literal is often itself structured and can be highly complex. Moreover, a value may have its own metadata, such as its creation date, information about ownership, permissions, and so on. Therefore, the Knora base ontology describes structured value types that can store this type of metadata. For example, in Figure \ref{fig:structured-values}, a book (\texttt{ex:book2}) has a title (identified by the predicate \texttt{ex:title}) and a publication date (\texttt{ex:pubdate}), each of which has some metadata.

\begin{figure}[h]
\centering

\begin{tikzpicture}
\node[vertex style=Gray] (resource) {ex:book2};

\node[vertex style=Orange, below of=resource,xshift=-10em,yshift=-4em] (title) {kb:TextValue}
edge [<-,cyan!60!blue] node[text style,above]{ex:title} (resource);

\node[vertex style=Orange, below of=resource,xshift=10em,yshift=-4em] (pubdate) {kb:DateValue}
edge [<-,cyan!60!blue] node[text style,above]{ex:pubdate} (resource);

\node[literal style=Red, below of=title,xshift=-6em,yshift=-6em] (titval) {\enquote{King Lear}}
edge [<-,cyan!60!green] node[text style,above]{kb:valueHasString} (title);

\node[literal style=Red, below of=title,xshift=6em,yshift=-6em] (crdate) {2015-08-12 13:00}
edge [<-,cyan!60!green] node[text style,above]{kb:creationDate} (title);

\node[literal style=Red, below of=pubdate,xshift=-6em,yshift=-6em] (startJDC) {2364669}
edge [<-,cyan!60!green] node[text style,above]{kb:valueHasStartJDC} (pubdate);

\node[literal style=Red, below of=pubdate,xshift=0em,yshift=-9em] (crdate2) {2015-08-12 13:03}
edge [<-,cyan!60!green] node[text style,above]{kb:creationDate} (pubdate);

\node[literal style=Red, below of=pubdate,xshift=6em,yshift=-6em] (endJDC) {2364669}
edge [<-,cyan!60!green] node[text style,above]{kb:valueHasEndJDC} (pubdate);

%\node[literal style=Red, right of=page,xshift=10em,yshift=-3em] (pagename) {a4r}
%edge [<-,cyan!60!blue] node[text style,above]{ex:pagename} (page);

%\node[vertex style=Blue, right of=sub2,xshift=10em,yshift=-3em] (sub3) {subject no. 3}
%edge [<-,cyan!60!green] node[text style,above]{predicate no. 5} (sub2);

\end{tikzpicture}

\caption{Structured values.}
\label{fig:structured-values}
\end{figure}

\subsection{Projects}

In Knora, each item of data belongs to some particular project. Each project using Knora must define a \texttt{kb:knoraProject}, which has these properties (cardinalities are indicated in parentheses after each property name):

\begin{description}
	\item[shortname (1)] A short name that can be used to identify the project in configuration files and the like.
	\item[basepath (1)] The filesystem path of the directory where the project's files are stored. \issue{192}{Should this be in the triplestore?}
	\item[foaf:name (0-1)] The name of the project.
	\item[description (0-1)] A description of the project.
	\item[belongsTo (0-1)] The \texttt{kb:Institution} that the project belongs to.
\end{description}

Resources and values are associated with a project by means of the \texttt{kb:attached\-To\-Project} property, as described in Section~\ref{sec:data-model}. Users are associated with a project by means of the \texttt{kb:is\-In\-Project} property, as described in Section~\ref{subsec:users-and-groups}.

\subsection{Resources}

\label{subsec:resources}

All the content produced by a project (e.g. digitised primary source materials or research data) must be stored in objects that belong to subclasses of \texttt{kb:Resource}, so that the Knora API server can query and update that content. Each project using the Knora base ontology must define its own OWL classes, derived from \texttt{kb:Resource}, to represent the types of data it deals with.

Resources have properties that point to different parts of the content they contain. For example, a resource representing a book could have a property called \texttt{hasAuthor}, pointing to the author of the book. There are two possible kinds of content in a Knora resource: Knora values (see Section~\ref{subsec:values}) or links to other resources (see Section~\ref{sec:links}). Properties that point to Knora values must be subproperties of \texttt{kb:hasValue}, and properties that point to other resources must be subproperties of \texttt{kb:hasLinkTo}. \issue{138}{Add a list of the Knora properties that projects are, and are not, allowed to use directly, and explain why.} Each property definition must specify the types that its subjects and objects must belong to (see Section~\ref{subsec:property-restrictions} for details).

Each project-specific resource class definition must use OWL cardinality restrictions to specify the properties that resources of that class can have (see Section~\ref{subsec:cardinalities} for details).

Resources are not versioned; only their values are versioned (see Section~\ref{subsec:values}).

A resource can be marked as deleted. An optional \texttt{kb:delete\-Comment} may be added to explain why the resource has been marked as deleted. Deleted resources are normally hidden. They cannot be undeleted, because even though resources are not versioned, it is necessary to be able to find out when a resource was deleted. If desired, a new resource can be created by copying data from a deleted resource.

\subsubsection{Properties of Resource}

\begin{description}
  \item[creationDate (1)] The time when the resource was created.
  \item[attachedToUser (1)] The user who owns the resource.
  \item[attachedToProject (1)] The project that the resource is part of.
  \item[lastModificationDate (0-1)] A timestamp indicating when the resource (or one of its values) was last modified.
  \item[seqnum (0-1)] The sequence number of the resource, if it is part of an ordered group of resources, such as the pages in a book. 
  \item[isDeleted (1)] Indicates whether the resource has been deleted.
  \item[deleteDate (0-1)] If the resource has been deleted, indicates when it was deleted.
  \item[deleteComment (0-1)] If the resource has been deleted, indicates why it was deleted.
\end{description}

Resources can have properties that point to other resources; see Section~\ref{sec:links}. A resource grants permissions to groups of users; see Section~\ref{sec:authorization}.

\subsubsection{Representations}

\label{subsubsec:representations}

It is not practical to store all data in RDF. In particular, RDF is not a good storage medium for binary data such as images. Therefore, Knora stores such data outside the triplestore, in ordinary files. A resource can have one or more files attached to it. For each file, there is a \texttt{kb:FileValue} in the triplestore containing metadata about the file (see Section~\ref{subsubsec:filevalue}). A resource that has file values must belong to of the subclasses of \texttt{kb:Representation}. The base class \texttt{Representation}, which is not intended to be used directly, has this property:

\begin{description}
	\item[hasFileValue (1-n)] Points to one or more file values.
\end{description}

Its subclasses, which are intended to be used directly in data, include:

\begin{description}
	\item[StillImageRepresentation] A representation containing still image files.
	\item[MovingImageRepresentation] A representation containing video files.
	\item[AudioRepresentation] A representation containing audio files.
	\item[DDDrepresentation] A representation containing 3D images.
	\item[TextRepresentation] A representation containing formatted text files, such as XML files.
	\item[DocumentRepresentation] A representation containing documents (such as PDF files) that are not text files.
\end{description}

There are two ways for a project to design classes for representations. The simpler way is to create a resource class that represents a thing in the world (such as \texttt{ex:Painting}) and also belongs to a subclass of \texttt{Representation}. This is adequate if the class can have only one type of file attached to it. For example, if paintings are represented only by still images, \texttt{ex:Painting} could be a subclass of \texttt{Still\-Image\-Representation}. This is the only approach supported in version 1 of the Knora API.

The more flexible approach, which is allowed by the Knora base ontology and will be supported by version 2 of the Knora API, is for each \texttt{ex:Painting} to use the \texttt{kb:hasRepresentation} property to point to other resources containing files that represent the painting. Each of these other resources can extend a different subclass of \texttt{Representation}. For example, a painting could have a \texttt{Still\-Image\-Representation} as well as a \texttt{DDDrepresentation}.

\subsection{Values}

\label{subsec:values}

The Knora base ontology defines a set of OWL classes that are derived from \texttt{kb:Value} and represent different types of structured values found in humanities data. This set of classes may not be extended by project-specific ontologies.

A value is always part of one particular resource, which points to it using some property derived from \texttt{hasValue}. For example, a project-specific ontology could specify a \texttt{Book} class with a property \texttt{hasSummary} (derived from \texttt{hasValue}), and that property could have a \texttt{knora-base:object\-Class\-Constraint} of \texttt{TextValue}. This would mean that the summary of each book is represented as a \texttt{TextValue}.

Knora values are versioned. Existing values are not modified. Instead, a new version of an existing value is created. The new version is linked to the old version via the \texttt{previousValue} property.

\enquote{Deleting} a value means marking it with \texttt{kb:is\-Deleted}. An optional \texttt{kb:delete\-Comment} may be added to explain why the value has been marked as deleted. Deleted values are normally hidden.

Most types of values are marked as deleted without creating a new version of the value. However, link values must be treated as a special case. Before a \texttt{LinkValue} can be marked as deleted, its reference count must be decremented to 0. Therefore, a new version of the \texttt{LinkValue} is made, with a reference count of 0, and it is this new version that is marked as deleted.

To simplify the enforcement of ontology constraints, and for consistency with resource updates, no new versions of a deleted value can be made; it is not possible to undelete. Instead, if desired, a new value can be created by copying data from a deleted value.

\subsubsection{Properties of Value}

\begin{description}
  \item[valueCreationDate (1)] The date and time when the value was created.
  \item[attachedToUser (1)] The user who owns the value.
  \item[attachedToProject (0-1)] The project that the value is part of. If not specified, defaults to the project of the containing resource.
  \item[valueHasString (1)] A human-readable string representation of the value's contents, which is available to Knora's full-text search index.
  \item[valueHasOrder (0-1)] A resource may have several properties of the same type with different values (which will be of the same class), and it may be necessary to indicate an order in which these values occur. For example, a book may have several authors which should appear in a defined order. Hence, \texttt{valueHasOrder}, when present, points to an integer literal indicating the order of a given value relative to the other values of the same property. These integers will not necessarily start at any particular number, and will not necessarily be consecutive.
  \item[previousValue (0-1)] The previous version of the value.
  \item[isDeleted (1)] Indicates whether the value has been deleted.
  \item[deleteDate (0-1)] If the value has been deleted, indicates when it was deleted.
  \item[deleteComment (0-1)] If the value has been deleted, indicates why it was deleted.
\end{description}

Each Knora value can grant permissions (see Section~\ref{sec:authorization}).

\subsection{Subclasses of Value}

\subsubsection{TextValue}

Represents text, possibly including markup. Property:

\begin{description}
	\item[valueHasStandoff (0-n)] Points to a Standoff node. See Section~\ref{sec:standoff}.
\end{description}

\subsubsection{DateValue}

Humanities data includes many different types of dates. In Knora, a date has a specified calendar, and is always represented as a period with start and end points (which may be equal), each of which has a precision (\texttt{DAY}, \texttt{MONTH}, or \texttt{YEAR}). Internally, the start and end points are stored as two Julian Day Numbers. This calendar-independent representation makes it possible to compare and search for dates regardless of the calendar in which they were entered. Properties:

\begin{description}
	\item[valueHasCalendar (1)] The name of the calendar in which the date should be displayed. Currently \texttt{GREGORIAN} and \texttt{JULIAN} are supported.
	\item[valueHasStartJDC (1)] The Julian Day Number of the start of the period (an \texttt{xsd:integer}).
	\item[valueHasStartPrecision (1)] The precision of the start of the period.
	\item[valueHasEndJDC (1)] The Julian Day Number of the end of the period (an \texttt{xsd:integer}).
	\item[valueHasEndPrecision (1)] The precision of the end of the period.
\end{description}

\subsubsection{IntValue}

Represents an integer. Property:

\begin{description}
	\item[valueHasInteger (1)] An \texttt{xsd:integer}.
\end{description}

\subsubsection{DecimalValue}

Represents an arbitrary-precision decimal number. Property:

\begin{description}
	\item[valueHasDecimal (1)] An \texttt{xsd:decimal}.
\end{description}

\subsubsection{TimeValue}

Represents a precise point on a timeline, e.g. relative to the beginning of an audio or video file. Property:

\begin{description}
	\item[valueHasTime (1)]	An \texttt{xsd:decimal} representing a number of seconds from the beginning of the timeline.
\end{description}

\subsubsection{GeomValue}

Represents a geometrical object as a JSON string, using normalized coordinates. Property:

\begin{description}
	\item[valueHasGeometry (1)] A JSON string.
\end{description}

\subsubsection{GeonameValue}

Represents a geolocation, using the numerical codes found at \url{http://geonames.org}. Property:

\begin{description}
	\item[valueHasGeonameCode (1)] the numerical code of a geographical feature from \url{http://geonames.org}, represented as an \texttt{xsd:integer}.
\end{description}

\subsubsection{IntervalValue}

Represents a time interval, with precise start and end times on a timeline, e.g. relative to the beginning of an audio or video file. Properties:

\begin{description}
	\item[valueHasIntervalStart (1)] An \texttt{xsd:decimal} representing the start of the interval in seconds.
	\item[valueHasIntervalEnd (1)]	An \texttt{xsd:decimal} representing the end of the interval in seconds.
\end{description}

\subsubsection{ListValue}

Projects often need to define lists or hierarchies of categories that can be assigned to many different resources. Then, for example, a user interface can provide a drop-down menu to allow the user to assign a category to a resource. The \texttt{ListValue} class provides a way to represent these sorts of data structures. It can represent either a flat list or a tree.

A \texttt{ListValue} has this property:

\begin{description}
	\item[valueHasListNode (1)] Points to the root \texttt{ListNode} of the list or tree.
\end{description}

Each \texttt{ListNode} can have the following properties:

\begin{description}
	\item[hasSubListNode (0-n)] Points to the node's child nodes, if any.
	\item[listNodePosition (1)] An integer indicating the node's position in the list of its siblings.
	\item[isRootNode (0-1)] Set to \texttt{true} if this is the root node.
	\item[listNodeName (0-n)] The node's human-readable name.
\end{description}

\subsubsection{FileValue}

\label{subsubsec:filevalue}

Knora stores certain kinds of data outside the triplestore, in files (see Section~\ref{subsubsec:representations}). Each digital object that is stored outside the triplestore has associated metadata, which is stored in the triplestore in a \texttt{kb:FileValue}. The base class \texttt{FileValue}, which is not intended to be used directly, has these properties:

\begin{description}
	\item[internalFilename (1)] The name of the file as stored by the Knora API server.
	\item[internalMimeType (1)] The MIME type of the file as stored by the Knora API server.
	\item[originalFilename (0-1)] The original name of the file when it was uploaded to the Knora API server.
	\item[originalMimeType (0-1)] The original MIME type of the file when it was uploaded to the Knora API server.
	\item[isPreview (0-1)] A boolean indicating whether the file is a preview, i.e.\ a small image representing the contents of the file. A preview is always a \texttt{Still\-Image\-File\-Value}, regardless of the type of the enclosing \texttt{Representation}.
\end{description}

The subclasses of \texttt{FileValue}, which are intended to be used directly in data, include:

\begin{description}
	\item[StillImageFileValue] Contains metadata about a still image file.
	\item[MovingImageFileValue] Contains metadata about a video file.
	\item[AudioFileValue] Contains metadata about an audio file.
	\item[DDDFileValue] Contains metadata about a 3D image file.
	\item[TextFileValue] Contains metadata about a text file.
	\item[DocumentFileValue] Contains metadata about a document (such as PDF) that is not a text file.
\end{description}

Each of these classes contains properties that are specific to the type of file it describes. For example, still image files have dimensions, video files have frame rates, and so on.

The files in a given representation must be semantically equivalent, meaning that coordinates that relate to one file must also be valid for other files in the same representation. Coordinates in Knora are expressed as fractions of the size of the object on some dimension; for example, image coordinates are expressed as fractions of its width and height, rather than in pixels. Therefore, the image files in a \texttt{StillImageRepresentation} must have the same aspect ratio, but they need not have the same dimensions in pixels. Similarly, the audio and video files in an \texttt{AudioRepresentation} or \texttt{MovingImageRepresentation} must have the same length in seconds, but may have different bitrates.

\texttt{FileValue} objects are versioned like other values, and the actual files stored by Knora are also versioned. Version 1 of the Knora API does not provide a way to retrieve a previous version of a file, but this feature will be added in a subsequent version of the API.

\subsubsection{LinkValue}

\label{subsubsec:linkvalue}

A \texttt{LinkValue} is an RDF \enquote{reification} containing metadata about a link between two resources. It is therefore a subclass of \texttt{rdf:Statement} as well as of \texttt{Value}. It has these properties:

\begin{description}
	\item[rdf:subject (1)] The resource that is the source of the link.
	\item[rdf:predicate (1)] The link property.
	\item[rdf:object (1)] The resource that is the target of the link.
	\item[valueHasRefCount (1)] The reference count of the link. This is meaningful when the \texttt{LinkValue} describes resource references in Standoff text markup (see Section~\ref{subsubsec:standoff-link}). Otherwise, the reference count will always be 1 (if the link exists) or 0 (if it has been deleted).
\end{description}

For details about how links are created in Knora, see Section~\ref{sec:links}.

\subsubsection{ExternalResValue}

Represents a resource that is not stored in the RDF triplestore managed by the Knora API server, but instead resides in an external repository managed by some other software. The \texttt{ExternalResValue} contains the information that the Knora API server needs in order to access the resource, assuming that a suitable gateway plugin is installed. \todo{Give more details on how this works.}

\begin{description}
	\item[extResAccessInfo (1)] The location of the repository containing the external resource (e.g. its URL).
	\item[extResId (1)] The repository-specific ID of the external resource.
	\item[extResProvider (1)] The name of the external provider of the resource.
\end{description}

\section{Links Between Resources}

\label{sec:links}

A link between two resources is expressed, first of all, as a triple, in which the subject is the resource that is the source of the link, the predicate is a \enquote{link property} (a subproperty of \texttt{kb:hasLinkTo}), and the object is the resource that is the target of the link.

It is also useful to store metadata about links. For example, Knora needs to know who owns the link, who has permission to modify it, when it was created, and so on. Such metadata cannot simply describe the link property, because then it would refer to that property in general, not to any particular instance in which that property is used to connect two particular resources. To attach metadata to a specific link in RDF, it is necessary to create an RDF \enquote{reification}. A reification makes statements about a particular triple (subject, predicate, object), in this case the triple that expresses the link between the resources. Knora uses reifications of type \texttt{kb:LinkValue} (described in Section~\ref{subsubsec:linkvalue}) to store metadata about links.

For example, suppose a project describes paintings that belong to collections. The project can define an ontology as follows (expressed here in Turtle format, and simplified for the purposes of illustration):

\begin{verbatim}
@prefix kb <http://www.knora.org/ontology/knora-base#> .
@prefix : <http://www.knora.org/ontology/paintings#> .

:Painting rdf:type owl:Class ;
    rdfs:subClassOf kb:Resource ,
        [ rdf:type owl:Restriction ;
            owl:onProperty :hasName ;
            owl:cardinality 1 ] ,
        [ rdf:type owl:Restriction ;
              owl:onProperty :title ;
              owl:cardinality 1 ] ;
        [ rdf:type owl:Restriction ;
              owl:onProperty :isInCollection ;
              owl:minCardinality 1 ] ;
        [ rdf:type owl:Restriction ;
              owl:onProperty :isInCollectionValue ;
              owl:minCardinality 1 ] .

:Collection rdf:type owl:Class ;
    rdfs:subClassOf kb:Resource ,
        [ rdf:type owl:Restriction ;
            owl:onProperty :hasName ;
            owl:cardinality 1 ] .
            
:hasName rdf:type owl:ObjectProperty ;
    rdfs:label "Name of artist" ;
    kb:subjectClassConstraint :Painting ;
    kb:objectClassConstraint kb:TextValue .
            
:title rdf:type owl:ObjectProperty ;
    rdfs:label "Title of painting"
    kb:subjectClassConstraint :Painting ;
    kb:objectClassConstraint kb:TextValue .

:hasName rdf:type owl:ObjectProperty ;
    rdfs:label "Name of collection" ;
    kb:subjectClassConstraint :Collection ;
    kb:objectClassConstraint kb:TextValue .
\end{verbatim}

To link the paintings to the collection, we must add a \enquote{link property} to the ontology. In this case, the link property will point from a painting to the collection it belongs to. Every link property must be a subproperty of \texttt{hasLinkTo}.

\begin{verbatim}
:isInCollection rdf:type owl:ObjectProperty ;
    rdfs:subPropertyOf kb:hasLinkTo ;
    kb:subjectClassConstraint :Painting ;
    kb:objectClassConstraint :Collection .
\end{verbatim}

We must then add a \enquote{link value property}, which will point from a painting to a \texttt{LinkValue} (described in Section~\ref{subsubsec:linkvalue}), which will contain metadata about the link between the property and the collection. In particular, the link value specifies the owner of the link, the date when it was created, and the permissions that determine who can view or modify it. The name of the link value property is constructed using a simple naming convention: the word \texttt{Value} is appended to the name of the link property. In this case, since our link property is called \texttt{isInCollectionValue}, the link value property must be called \texttt{ex:isOnPageValue}. Every link value property must be a subproperty of \texttt{kb:hasLinkToValue}.

\begin{verbatim}
:isInCollectionValue rdf:type owl:ObjectProperty ;
    rdfs:subPropertyOf kb:hasLinkToValue ;
    kb:subjectClassConstraint :painting ;
    kb:objectClassConstraint kb:LinkValue .
\end{verbatim}

Given this ontology, we can create some RDF data describing a painting and a collection:

\begin{verbatim}
@prefix paintings <http://www.knora.org/ontology/paintings#> .
@prefix data <http://www.knora.org/ontology/paintings/data#> .

data:dali_4587 rdf:type paintings:Painting ;
    paintings:title data:value_A ;
    paintings:hasName data:value_B .

data:value_A rdf:type kb:TextValue ;
    kb:valueHasString "The Persistence of Memory" .
                
data:value_B rdf:type kb:TextValue ;
    kb:valueHasString "Salvador Dali" .
                
data:pompidou rdf:type paintings:Collection ;
    paintings:hasName data:value_C .
                
data:value_C rdf:type kb:TextValue ;
    kb:valueHasString "Centre Pompidou, Paris" .
\end{verbatim}

We can then state that the painting is in the collection:

\begin{verbatim}
data:dali_4587 paintings:isInCollection data:pompidou ;
    paintings:isinCollectionValue data:value_D .

data:value_D rdf:type kb:LinkValue ;
    rdf:subject data:dali_4587 ;
    rdf:predicate paintings:isInCollection ;
    rdf:object data:pompidou ;
    kb:valueHasRefCount 1 .
\end{verbatim}

This creates a link (\texttt{isInCollection}) between the painting and the collection, along with a reification containing metadata about the link. We can visualise the result as the graph shown in Figure~\ref{fig:basic-link-graph}.

\begin{figure}[h]
\centering
\begin{tikzpicture}
\node[vertex style=Gray] (dali) {dali\_4587};

\node[vertex style=Orange, below of=dali,xshift=-10em,yshift=-3em] (artist) {data:value\_B}
edge [<-,cyan!60!blue] node[text style,above]{hasName} (dali);
\node[literal style=Red, below of=artist,xshift=0em,yshift=-6em] (artistlit) {\enquote{Salvador Dali}}
edge [<-,cyan!60!green] node[text style,above]{valueHasString} (artist);

\node[vertex style=Orange, below of=dali,xshift=-0em,yshift=-3em] (title) {data:value\_A}
edge [<-,cyan!60!blue] node[text style,above]{hasName} (dali);
\node[literal style=Red, below of=title,xshift=0em,yshift=-6em] (titlelit) {\enquote{The Persistence of Memory}}
edge [<-,cyan!60!green] node[text style,above]{valueHasString} (title);

\node[vertex style=Gray, right of=dali,xshift=14em] (pompidou) {pompidou}
edge [<-,cyan!60!green] node[text style,above]{isInCollection} (dali);

\node[vertex style=Orange, below of=pompidou,xshift=0em,yshift=-3em] (collname) {data:value\_C}
edge [<-,cyan!60!blue] node[text style,above]{hasName} (pompidou);
\node[literal style=Red, below of=collname,xshift=0em,yshift=-6em] (collnamelit) {\enquote{Centre Pompidou}}
edge [<-,cyan!60!green] node[text style,above]{valueHasString} (collname);

\node[vertex style=Orange, right of=title,xshift=6em,yshift=0em] (link) {value\_D}
edge [->,cyan!60!green, bend left=10] node[text style,above]{subject} (dali)
edge [->,cyan!60!green] node[text style,above]{object} (pompidou)
edge [->,cyan!60!green] node[text style,above]{predicate} (3,0);

\path (dali) edge [->,cyan!60!green, bend left=10] node[text style,above]{isInCollectionValue} (link);

\node[literal style=Red, below of=link,xshift=0em,yshift=-9em] (refcnt) {1}
edge [<-,cyan!60!green] node[text style,above]{valueHasRefCount} (link);

\end{tikzpicture}
\caption{An RDF graph showing how two Knora resources are linked together.}
\label{fig:basic-link-graph}
\end{figure}

\section{Text with Standoff Nodes}

\label{sec:standoff}

Knora is designed to be able to store text with markup, which can indicate formatting and structure, as well as the complex observations involved in transcribing handwritten manuscripts. One popular way of representing text in the humanities is to encode it as TEI/XML\footnote{TEI refers both to an organization and an XML-based markup language (or more precisely: a set of grammar modules -- XML schemas -- that can be combined to define a markup language). For reasons of clarity, we use the term TEI/XML to refer to the markup language.} using the \href{http://www.tei-c.org/index.xml}{Text Encoding Initiative (TEI)} guidelines~\cite{P5}. In Knora, a TEI/XML document can be stored as a file with attached metadata.

However, Knora also supports \enquote{standoff} nodes, which are stored separately from the text. This has some advantages over embedded markup such as XML.\footnote{It is also possible to encode standoff markup using XML. For example, the TEI discusses standoff markup in its guidelines~\cite[chapters 16.9 and 20.4]{P5}. However, standoff markup is not widely applied in the TEI community. The main focus lies on encoding a hierarchy of elements.} While XML requires markup to have a hierarchical structure, and does not allow overlapping tags, standoff nodes do not have these limitations~\cite{Schmidt_Standoff}. A \enquote{standoff node} can assign an attribute to any substring in the text by giving its start and end positions.\footnote{Unlike in corpus linguistics, we do not use any tokenization resulting in a form of predefined segmentation that would limit the user's possibility to freely annotate any ranges in the text.} For example, suppose we have the following text:

\begin{quote}
This \textit{sentence \textbf{has overlapping}}\textbf{ visual} attributes.
\end{quote}

This would require just two standoff nodes: \texttt{(italic, start=5, end=29)} and \texttt{(bold, start=14, end=36)}.

Moreover, standoff makes it possible to mark up the same text in different, possibly incompatible ways, allowing for different interpretations without making redundant copies of the text. In the Knora base ontology, any text value can have standoff nodes.

By representing standoff as RDF triples, Knora makes markup searchable across multiple text documents in a repository. For example, if a repository contains documents in which references to persons are indicated in standoff, it is straightforward to find all the documents mentioning a particular person. Knora's standoff support is intended to make it possible to convert documents with embedded, hierarchical markup, such as TEI/XML, into RDF standoff and back again, with no data loss, thus bringing the benefits of RDF to existing TEI-encoded documents (cf.~\cite[3]{Schmidt_Standoff}).

In the Knora base ontology, a \texttt{TextValue} can have one or more standoff nodes. Each standoff node indicates the start and end positions of a substring in the text that has a particular attribute. The OWL class \texttt{kb:Standoff}, which is the base class of all standoff node classes, has these properties:

\begin{description}
	\item[standoffHasAttribute (1)] The name of the attribute.
	\item[standoffHasStart (1)] The index of the first character in the text that has the attribute.
	\item[standoffHasEnd (1)] The index of the last character in the text that has the attribute, plus 1.
\end{description}

The \texttt{Standoff} class is not used directly in RDF data; instead, its subclasses are used. A few subclasses are currently provided, and more will be added to support TEI semantics.

\subsection{Subclasses of Standoff}

\subsubsection{StandoffVisualAttribute}

Represents a formatting attribute such as {\it boldface} or {\bf italics}. The value of \texttt{standoffHasAttribute} is the name of the formatting attribute, and can be any string. It is up to the text renderer to interpret these names.

\subsubsection{StandoffHref}

Indicates that a substring is associated with a resource on the Internet, i.e.\ a URL. It has this property:

\begin{description}
	\item[standoffHasHref (1)] An \texttt{xsd:anyURI} representing the URL.
\end{description}

\subsubsection{StandoffLink}

\label{subsubsec:standoff-link}

Indicates that a substring is associated with a Knora resource. For example, if a repository contains resources representing persons, a text could be marked up so that each time a person's name is mentioned, a \texttt{StandoffLink} connects the name to the Knora resource describing that person. It has this property:

\begin{description}
	\item[standoffHasLink (1)] The resource that the link points to.
\end{description}

One of the design goals of the Knora ontology is to make it easy and efficient to find out which resources contain references to a given resource. Direct links are easier and more efficient to query than indirect links. Therefore, when a text value contains a resource reference in its standoff nodes, there must also be a direct link between the containing resource and the target resource, along with an RDF reification (a \texttt{kb:LinkValue}) describing the link, as discussed in Section~\ref{sec:links}. In this case, the link property is always \texttt{kb:hasStandoffLinkTo}, and the link value property (which points to the \texttt{LinkValue}) is always \texttt{kb:hasStandoffLinkToValue}.

The Knora API server automatically creates and updates direct links and reifications for standoff resource references when it creates and updates text values. To do this, it keeps track of the number of text values in each resource that contain at least one standoff reference to a given target resource. It stores this number as the reference count of the \texttt{LinkValue} (see Section~\ref{subsubsec:linkvalue}) describing the direct link. Each time this number changes, it makes a new version of the \texttt{LinkValue}, with an updated reference count. When the reference count reaches zero, it removes the direct link and makes a new version of the \texttt{LinkValue}, marked with \texttt{kb:isDeleted}.

For example, if \texttt{data:R1} is a resource with a text value in which the resource \texttt{data:R2} is referenced, the repository could contain the following triples:

\begin{verbatim}
data:R1 ex:hasComment data:V1 .

data:V1 rdf:type kb:TextValue ;
    kb:valueHasString "This link is internal." ;
    kb:valueHasStandoff data:SO1 .

data:SO1 rdf:type kb:StandoffLink ;
    kb:standoffHasStart: 5 ;
    kb:standoffHasEnd: 9 ;
    kb:standoffHasLink data:R2 .

data:R1 kb:hasStandoffLinkTo data:R2 .
data:R1 kb:hasStandoffLinkToValue data:LV1 . 

data:LV1 rdf:type kb:LinkValue ;            
    rdf:subject data:R1 ;
    rdf:predicate kb:hasStandoffLinkTo ;
    rdf:object data:R2 ;
    kb:valueHasRefCount 1 .
\end{verbatim}

Figure~\ref{fig:standoff-link-graph} illustrates the result.

Link values created automatically for resource references in standoff are automatically visible to all users, as long as they have permission to see the source and target resources. The owner of these link values is always \texttt{kb:SystemUser} (see Section \ref{subsec:users-and-groups}).

\begin{figure}[h]
\centering

\begin{tikzpicture}

\node[vertex style=Gray] (r1) {R1};

\node[vertex style=Orange, below of=r1,xshift=-10em,yshift=-3em] (hasComment) {V1}
edge [<-,cyan!60!blue] node[text style,above]{ex:hasComment} (r1);
\node[literal style=Red, below of=artist,xshift=0em,yshift=-6em] (string) {\enquote{This link is internal.}}
edge [<-,cyan!60!green] node[text style,above]{valueHasString} (hasComment);

\node[vertex style=Green, below of=r1,xshift=3em,yshift=-4em] (so1) {SO1}
edge [<-,cyan!60!blue] node[text style,above]{valueHasStandoff} (hasComment);
\node[literal style=Red, below of=so1,xshift=-3em,yshift=-8em] (start) {5}
edge [<-,cyan!60!green] node[text style,above]{standoffHasStart} (so1);
\node[literal style=Red, below of=so1,xshift=3em,yshift=-8em] (end) {9}
edge [<-,cyan!60!green] node[text style,above]{standoffHasEnd} (so1);

\node[vertex style=Gray, right of=r1,xshift=16em] (r2) {R2}
edge [<-,cyan!60!green] node[text style,above]{hasStandoffLinkTo} (r1)
edge [<-,cyan!60!green, bend left=22] node[text style,above]{standoffHasLink} (so1);

\node[vertex style=Orange, right of=so1,xshift=10em,yshift=0em] (link) {LV1}
edge [->,cyan!60!green, bend left=10] node[text style,above]{subject} (r1)
edge [->,cyan!60!green] node[text style,above]{object} (r2)
edge [->,cyan!60!green] node[text style,above]{predicate} (4,0);

\path (r1) edge [->,cyan!60!green, bend left=10] node[text style,above]{hasStandoffLinkToValue} (link);

\node[literal style=Red, below of=link,xshift=6em,yshift=-6em] (refcnt) {1}
edge [<-,cyan!60!green] node[text style,above]{valueHasRefCount} (link);

\end{tikzpicture}

\caption{An RDF graph showing a link resulting from a reference to a resource in a standoff node.}
\label{fig:standoff-link-graph}
\end{figure}

\section{Authorization}

\label{sec:authorization}

\subsection{Users and Groups}

\label{subsec:users-and-groups}

Each Knora user is represented by an object belonging to the class \texttt{kb:User}, which is a subclass of \texttt{foaf:Person}, and has the following properties:

\begin{description}
	\item[userid (1)] A unique identifier that the user must provide when logging in.
	\item[password (1)] A cryptographic hash of the user's password.
	\item[email (0-n)] Email addresses belonging to the user.
	\item[isInProject (0-n)] Projects that the user is a member of.
	\item[isInGroup (0-n)] Project-specific groups that the user is a member of.
	\item[foaf:familyName (1)] The user's family name.
	\item[foaf:givenName (1)] The user's given name.
\end{description}

Knora's concept of access control is that an object (a resource or value) can grant permissions to groups of users (but not to individual users). There are four built-in groups:

\begin{description}
	\item[UnknownUser] Any user who has not logged into the Knora API server is automatically assigned to this group.
	\item[KnownUser] Any user who has logged into the Knora API server is automatically assigned to this group.
	\item[ProjectMember] When checking a user's permissions on an object, the user is automatically assigned to this group if she is a member of the project that the object belongs to.
	\item[Owner] When checking a user's permissions on an object, the user is automatically assigned to this group if he is the owner of the object.
\end{description}

A project-specific ontology can define additional groups, which must belong to the OWL class \texttt{kb:UserGroup}.

There is one built-in \texttt{SystemUser}, which is the owner of link values created automatically for resource references in standoff markup (see Section~\ref{subsubsec:standoff-link}).

\subsection{Permissions}

The owner of an object is always allowed to perform any operation on it. An object can grant the following permissions, which are stored in a compact format in a single string, which is the object of the predicate 
\texttt{kb:hasPermissions}:

\begin{enumerate}
	\item \textbf{Restricted view permission (\texttt{RV})} Allows a restricted view of the object, e.g.\ a view of an image with a watermark.
	\item \textbf{View permission (\texttt{V})} Allows an unrestricted view of the object. Having view permission on a resource only affects the user's ability to view information about the resource other than its values. To view a value, she must have view permission on the value itself.
	\item \textbf{Modify permission (\texttt{M})} For values, this permission allows a new version of a value to be created. For resources, this allows the user to create a new value (as opposed to a new version of an existing value), or to change information about the resource other than its values. When he wants to make a new version of a value, his permissions on the containing resource are not relevant. However, when he wants to change the target of a link, the old link must be deleted and a new one created, so he needs modify permission on the resource.
	\item \textbf{Delete permission (\texttt{D})} Allows the item to be marked as deleted.
	\item \textbf{Change rights permission (\texttt{CR})} Allows the permissions granted by the object to be changed.
\end{enumerate}

Each permission in the above list implies all lower-numbered permissions. A user's permission level on a particular object is calculated in the following way:

\begin{enumerate}
	\item Make a list of the groups that the user belongs to, including \texttt{Owner} and/or \texttt{ProjectMember} if applicable.
	\item If the user is the owner of the object, give her the highest level of permissions.
	\item Otherwise, make a list of the permissions that she can obtain on the object, by iterating over the permissions that the object grants. For each permission, if she is in the specified group, add the specified permission to the list of permissions she can obtain.
    \item From the resulting list, select the highest-level permission.
    \item If the result is that she would have no permissions, give her whatever permission \texttt{UnknownUser} would have.
\end{enumerate}

To view a link between resources, a user needs permission to view the source and target resources. He also needs permission to view the \texttt{LinkValue} representing the link, unless the link property is \texttt{hasStandoffLinkTo} (see \ref{subsubsec:standoff-link}).

The format of the object of \texttt{kb:hasPermissions} is as follows:

\begin{itemize}
	\item Each permission is represented by the one-letter or two-letter abbreviation given above.
	\item Each permission abbreviation is followed by a space, then a comma-separated list of groups that the permission is granted to.
	\item The IRIs of built-in groups are shortened using the \texttt{knora-base} prefix.
	\item Multiple permissions are separated by a vertical bar (\texttt{|}).
\end{itemize}

For example, if an object grants view permission to unknown and known users, and modify permission to project members, the resulting permission literal would be:

\begin{verbatim}
V knora-base:UnknownUser,knora-base:KnownUser|M knora-base:ProjectMember
\end{verbatim}

\section{Consistency Checking}

Knora tries to enforce repository consistency by checking constraints that are specified in the Knora base ontology and in project-specific ontologies. Two types of constraints are enforced:

\begin{itemize}
	\item Cardinalities in OWL class definitions.
	\item Constraints on the types of the subjects and objects of OWL object properties.
\end{itemize}

The implementation of consistency checking is partly triplestore-dependent; Knora may be able to provide stricter checks with some triplestores than with others.

\subsection{OWL Cardinalities}

\label{subsec:cardinalities}

As noted in Section~\ref{subsec:resources}, each subclass of \texttt{Resource} must use OWL cardinality restrictions to specify the properties it can have. More specifically, a resource is allowed to have a property that is a subproperty of \texttt{kb:hasValue} or \texttt{kb:hasLinkTo} only if the resource's class has some cardinality for that property. In addition, Knora supports, and attempts to enforce, the following cardinality constraints:
 
\begin{description}
	\item[owl:cardinality 1] A resource of this class must have exactly one instance of the specified property.
	\item[owl:minCardinality 1] A resource of this class must have at least one instance of the specified property.
	\item[owl:maxCardinality 1] A resource of this class may have zero or one instance of the specified property.
	\item[owl:minCardinality 0] A resource of this class may have zero or more instances of the specified property.
\end{description}

For more information about OWL cardinalities, see~\cite[§2.1, Object Property Restrictions]{OWL_2_Quick_Reference_Guide}.

\subsection{Constraints on the Types of Property Subjects and Objects}

\label{subsec:property-restrictions}

When a project-specific ontology defines a property, it must indicate the types that are allowed as subjects and objects of the property. This is done using the following Knora-specific properties:

\begin{description}
	\item[subjectClassConstraint] Specifies the class that subjects of the property must belong to. Knora will attempt to enforce this constraint.
	\item[objectClassConstraint] If the property is an object property, specifies the class that objects of the property must belong to. Knora will attempt to enforce this constraint.
	\item[objectDatatypeConstraint] If the property is a datatype property, specifies the type of literals that can be objects of the property. Knora will not attempt to enforce this constraint, but it is useful for documentation purposes.
\end{description}

\subsection{Consistency Constraint Example}

A project-specific ontology could define consistency constraints as in this simplified example:

\begin{verbatim}
:book rdf:type owl:Class ;
    rdfs:subClassOf knora-base:Resource ,
        [ rdf:type owl:Restriction ;
          owl:onProperty :hasTitle ;
          owl:cardinality "1"^^xsd:nonNegativeInteger ] ,
        [ rdf:type owl:Restriction ;
          owl:onProperty :hasAuthor ;
          owl:minCardinality "0"^^xsd:nonNegativeInteger ] .

:hasTitle rdf:type owl:ObjectProperty ;
    knora-base:subjectClassConstraint :book ;
    knora-base:objectClassConstraint knora-base:TextValue .

:hasAuthor rdf:type owl:ObjectProperty ;
    knora-base:subjectClassConstraint :book ;
    knora-base:objectClassConstraint knora-base:TextValue .
\end{verbatim}

\section{Open Questions}

\subsection{Extending Existing Resource Definitions}

How should extensions of existing resources be handled? Project B extends a resource defined in the project A ontology, by adding new properties/values which are interesting for project B.

\printbibliography

\end{document}  
